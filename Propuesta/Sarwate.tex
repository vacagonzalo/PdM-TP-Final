\documentclass[
12pt,
spanish,
singlespacing,
parskip,
headsepline]{article}
\usepackage[spanish]{babel}
\usepackage{palatino}
\usepackage{graphicx}
\usepackage{booktabs}
\usepackage{listings}
\lstset
{ %Formatting for code in appendix
	language=Python,
	basicstyle=\footnotesize,
	numbers=left,
	stepnumber=1,
	showstringspaces=false,
	tabsize=4,
	breaklines=true,
	breakatwhitespace=false,
}
\title{Algoritmo de Sarwate}
\author{Gonzalo Nahuel Vaca}

\begin{document}
	
\maketitle

\section{Introducción}
\label{sec:intro}

El algoritmo de Sarwate permite realizar el cálculo de redundancia cíclica (CRC) operando byte a byte.
Esto es una mejora frente al método tradicional de realizar una división contra un polinomio que obliga a analizar un bit por ciclo de máquina.

Para este trabajo se propone realizar un CRC de 8 bits.

\subsection{Plataforma de desarrollo}
\label{sub:plataforma}

Se utilizará la placa \emph{NUCLEO-F429ZI} y el RTOS \emph{Mbed}.

\section{Máquina de estados finitos}
\label{sec:mef}

Se plantea una solución como una máquina de estados finitos de tipo \emph{Mealy}.
En el cuadro \ref{tab:estados} se puede observar la representación de las transiciones entre los 256 estados.

\begin{table}[!htbp] \centering
\begin{tabular}{|c|c|c|c|c|}
\hline
$S_{actual} \backslash S_{siguiente}$ & $S_1$             & $S_2$             & $\cdots$ & $S_{256}$               \\ \hline
$S_1$                                 & $I_{1,1} / O_1$   & $I_{1,2} / O_2$   & $\cdots$ & $I_{1,256} / O_{256}$   \\ \hline
$S_2$                                 & $I_{2,1} / O_1$   & $I_{2,2} / O_2$   & $\cdots$ & $I_{2,256} / O_{256}$   \\ \hline
$\vdots$                              & $\vdots$          & $\vdots$          & $\ddots$ & $\vdots$                \\ \hline
$S_{256}$                             & $I_{256,1} / O_1$ & $I_{256,2} / O_2$ & $\cdots$ & $I_{256,256} / O_{256}$ \\ \hline
\end{tabular}
\caption{transición de estados}
\label{tab:estados}
\end{table}

Los estados se expresan como $S_n$, las entradas como $I_{i,j}$ y las salidas $O_{n}$.
El estado inicial es $S_1$ con una salida $O_1$ que en todo caso será igual a cero.

\section{Funcionamiento}
\label{sec:func}

El algoritmo funciona de la siguiente manera:

\begin{enumerate}
	\item Se crea una tabla con los 256 resultados posibles de la división polinómica.
	\item Se comienza cargando como salida el primer elemento de la tabla.
	\item Se realiza una operación \emph{XOR} con la salida actual y el primer byte de la secuencia.
	\item El resultado de la operación el es índice (estado) de la tabla que tiene el valor de la nueva salida.
	\item Se opera hasta finalizar los datos a verificar.
	\item La última salida obtenida es el CRC del paquete.
	\item El CRC calculado debe coincidir con el CRC transmitido.
\end{enumerate}

\section{Codificación}
\label{sec:cod}

\subsection{Obtención de tabla}
\label{sub:tabla}

Se decidió de manera arbitraria usar el siguiente polinomio: 

$$0x^{7} + 0x^{6} + 0x^{5} + x^{4} + x^{3} + x^{2} + 0x^{1} + 0x $$

Simplificado: 

$$x^{4} + x^{3} + x^{2}$$

Si solo tomamos los coeficientes, obtenemos el número hexadecimal $1C$.

Como se mencionó en la sección \ref{sec:intro}, la tabla guarda los resultados de la división del polinomio con los 256 valores posibles de un byte.
Se muestra el siguiente código para lograrlo: 

\begin{lstlisting}[language=Python]
POLYNOMIAL = 0x1C

def buildTable(poly=POLYNOMIAL) -> list:
	length = 256
	polynomial = poly
	mask = 0xFF
	table = [0] * length
	for i in range(length):
		crc = i
		for _ in range(8):
			if(crc & 0x80):
			crc = (crc << 1) ^ polynomial
			else:
			crc = crc << 1
		table[i] = (crc & mask)
	return table
\end{lstlisting}

Se arroja la siguiente tabla de valores:

\begin{lstlisting}[language=Python]
TABLE = [
	0, 28, 56, 36, 112, 108, 72, 84, 224, 252, 216, 196, 144, 140, 168, 180, 220,
	192, 228, 248, 172, 176, 148, 136, 60, 32, 4, 24, 76, 80, 116, 104, 164,
	184, 156, 128, 212, 200, 236, 240, 68, 88, 124, 96, 52, 40, 12, 16, 120,
	100, 64, 92, 8, 20, 48, 44, 152, 132, 160, 188, 232, 244, 208, 204, 84,
	72, 108, 112, 36, 56, 28, 0, 180, 168, 140, 144, 196, 216, 252, 224, 136,
	148, 176, 172, 248, 228, 192, 220, 104, 116, 80, 76, 24, 4, 32, 60, 240, 236,
	200, 212, 128, 156, 184, 164, 16, 12, 40, 52, 96, 124, 88, 68, 44, 48, 20, 8,
	92, 64, 100, 120, 204, 208, 244, 232, 188, 160, 132, 152, 168, 180, 144, 140,
	216, 196, 224, 252, 72, 84, 112, 108, 56, 36, 0, 28, 116, 104, 76, 80, 4, 24,
	60, 32, 148, 136, 172, 176, 228, 248, 220, 192, 12, 16, 52, 40, 124, 96, 68, 88,
	236, 240, 212, 200, 156, 128, 164, 184, 208, 204, 232, 244, 160, 188, 152, 132,
	48, 44, 8, 20, 64, 92, 120, 100, 252, 224, 196, 216, 140, 144, 180, 168, 28, 0,
	36, 56, 108, 112, 84, 72, 32, 60, 24, 4, 80, 76, 104, 116, 192, 220, 248, 228,
	176, 172, 136, 148, 88, 68, 96, 124, 40, 52, 16, 12, 184, 164, 128, 156, 200, 212,
	240, 236, 132, 152, 188, 160, 244, 232, 204, 208, 100, 120, 92, 64, 20, 8, 44, 48
]
\end{lstlisting}

\subsection{Obtención del CRC}
\label{sub:obtencion}

Para obtener el CRC se ejecuta la máquina de estados finitos explicada en la sección \ref{sec:mef}.
Se propone el siguiente código:

\begin{lstlisting}[language=Python]
def buildCRC8(data, table=TABLE) -> int:
	crc8: int = table[0]
	indx: int = 0
	for byte in data:
		indx = crc8 ^ byte
		crc8 = table[indx]
	return crc8
\end{lstlisting}

En el código, el estado $S_n$ está representada por la variable \emph{indx}, la salida por \emph{crc8} y la entrada por \emph{byte}.

Finalmente se puede verificar el CRC invocando a la máquina de estados en el \emph{payload} y comparando el CRC transmitido.
Como se muestra en el siguiente código:

\begin{lstlisting}[language=Python]
def computeCRC8(frame, table=TABLE) -> bool:
	if len(frame) < 2:
		return False
	data: list = frame[0:-1]
	crc8: str = frame[-1]
	comp: str = buildCRC8(data, table)
	return crc8 == comp
\end{lstlisting}

\subsection{Codificación para microcontroladores}
\label{sub:langc}

Creación de la tabla:

\begin{lstlisting}[language=C]
#define TABLE_LENGTH 256
#define OCTET_MASK 0X000000FF

unsigned char CRC8TABLE[TABLE_LENGTH];

void GenerateTable(void)
{
	unsigned char crcValue;
	unsigned char polynomial = 0x1C;
	unsigned int i, j;
	
	for(i=0; i < TABLE_LENGTH; i++)
	{
		crcValue = i;
		for(j=0; j < 8; j++)
		{
			if(crcValue & 0x80) {
				crcValue = (crcValue << 1);
				crcValue = crcValue ^ polynomial;
			}
			else {
				crcValue = crcValue << 1;
			}
		}
		CRC8TABLE[i] = crcValue & OCTET_MASK;
	}
}
\end{lstlisting}

Cálculo del CRC:

\begin{lstlisting}[language=C]
unsigned char buildCRC8(unsigned int* bufferPtr, unsigned int bufferLenght)
{
	unsigned int i;
	unsigned char crcValue = 0x00;
	unsigned int tableIndex;
	
	for(i=0; i < bufferLenght; i++)
	{
		tableIndex = crcValue^(bufferPtr[i] & 0xFF);
		crcValue = CRC8TABLE[tableIndex];
	}
	return crcValue;
}

\end{lstlisting}

\end{document}